\documentclass[final,hyperref={pdfpagelabels=false}]{beamer}
\usepackage{grffile}
\mode<presentation>{\usetheme{PI1}}
\usepackage[portuges, brazil]{babel}   
\usepackage[utf8]{inputenc}
\usepackage{amsmath,amsthm, amssymb, latexsym}
\boldmath
\usepackage[orientation=portrait,size=a0,scale=1.4,debug]{beamerposter}
\usepackage{array,booktabs,tabularx}
\newcolumntype{Z}{>{\centering\arraybackslash}X} % centered tabularx columns
\newcommand{\pphantom}{\textcolor{ta3aluminium}} % phantom introduces a vertical space in p formatted table columns??!!

\listfiles

%%%%%%%%%%%%%%%%%%%%%%%%%%%%%%%%%%%%%%%%%%%%%%%%%%%%%%%%%%%%%%%%%%%%%%%%%%%%%%%%%%%%%%
\graphicspath{{images/}}

% -------------------------- TÍTULO ------------------------------
\title{\Huge \textbf{A Queda de Troia}}
\vspace{150cm} % Espaço vertical de 50cm
\author{Bruno Silva Souza, Cauã Mendes Gomes, Keila Suellen Santos Sacramento, Lucas de Sá Magalhães e Lucas Pinheiro da Costa Pedrosa.}
\vspace{1cm} % Espaço vertical de 1cm
\institute[SENAC]{Centro Universitário Senac - Campus Santo Amaro}

%%%%%%%%%%%%%%%%%%%%%%%%%%%%%%%%%%%%%%%%%%%%%%%%%%%%%%%%%%%%%%%%%%%%%%%%
\newlength{\columnheight}
\setlength{\columnheight}{105cm}

\begin{document}
\begin{frame}
    \vspace{1cm}
    \begin{columns}
        % ---------------------------------------------------------%
        % Primeira coluna - Introdução e Metodologia
        \begin{column}{.49\textwidth}
            \begin{beamercolorbox}[center,wd=\textwidth]{postercolumn}
                \begin{minipage}[T]{.85\textwidth} 
                    \parbox[t][\columnheight]{\textwidth}{ 
                        % TÍTULO INTRODUÇÃO
                        \begin{block}{\textbf{Introdução}}
                            A Queda de Troia é um jogo 2D inspirado no épico filme de 2004, permitindo aos jogadores reviverem a lendária Guerra de Troia ao assumirem papéis de heróis como Aquiles, Menelau e Heitor. O jogo combina narrativa, batalhas épicas e interatividade para ensinar sobre a história e mitologia da antiga Grécia.
                        \end{block}
                        
                        % TÍTULO METODOLOGIA
                        \begin{block}{\textbf{Fases do Jogo}}
                            \begin{itemize}
                                \item \textbf{Fase 1:} Paris sequestra Helena e a leva para Troia, enquanto os gregos iniciam uma tentativa de resgatá-la.
                                \item \textbf{Fase 2:} O jogador controla Heitor em uma batalha contra Menelau, defendendo a honra e segurança de Troia.
                                \item \textbf{Fase 3:} Os gregos constroem o Cavalo de Troia para invadir a cidade. O jogador deve resolver um quebra-cabeça estratégico para concluir a montagem.
                                \item \textbf{Fase 4:} Em meio à destruição de Troia, o jogador enfrenta obstáculos enquanto tenta escapar. A decisão final determinará o destino da cidade: sobreviver ou sucumbir.
                            \end{itemize}
                        \end{block}
                        
                        % TÍTULO PERSONAGENS
                       \begin{block}{\textbf{Personagens}}
    \begin{itemize}
      \item \begin{minipage}[t]{0.2\textwidth}
                \centering
                 \textbf{Aquiles}
                 \vspace{1cm} % Espaço vertical de 1cm
                \includegraphics[width=0.3\textwidth]{aquiles.png} \\
              \end{minipage} 
              O maior guerreiro grego, destemido e habilidoso.
        \vspace{1cm} % Espaço vertical de 1cm
        \item \begin{minipage}[t]{0.2\textwidth}
                \centering
                \hspace{4cm} % Espaço de 2cm à esquerda
                \includegraphics[width=0.3\textwidth]{menelau.png} \\
                \textbf{Menelau}
              \end{minipage} 
              O rei de Esparta, motivado pela vingança lidera os gregos na vigança contra tróia
        \vspace{1cm} % Espaço vertical de 1cm
        \item \begin{minipage}[t]{0.2\textwidth}
                \centering
                \textbf{Helena}
                \includegraphics[width=0.3\textwidth]{helena.png} \\
              \end{minipage} 
             Rainha de Esparta, cuja beleza e paixão desencadeiam a guerra de Tróia.
        \vspace{1cm} % Espaço vertical de 1cm
        \item \begin{minipage}[t]{0.2\textwidth}
                \centering
                \textbf{Heitor}
                \includegraphics[width=0.4\textwidth]{heitor.png} \\
              \end{minipage} 
             Herói troiano, irmão de Paris e defensor de Tróia.
        \vspace{1cm} % Espaço vertical de 1cm
        \item \begin{minipage}[t]{0.2\textwidth}
                \centering
                \textbf{Paris}
                \includegraphics[width=0.4\textwidth]{paris.png} 
              \end{minipage} 
              Príncipe de Tróia e responsável pelo início da guerra ao levar Helena para sua cidade. 
    \end{itemize}
\end{block}

                    }
                \end{minipage}
            \end{beamercolorbox}
        \end{column}
        % ---------------------------------------------------------%

        % Segunda coluna - Jogabilidade, Tecnologias e Imagem
        \begin{column}{.49\textwidth}
            \begin{beamercolorbox}[center,wd=\textwidth]{postercolumn}
                \begin{minipage}[T]{.95\textwidth} 
                    \parbox[t][\columnheight]{\textwidth}{ 
                        % TÍTULO JOGABILIDADE
                        \begin{block}{\textbf{Jogabilidade}}
                            \begin{itemize}
                                \item Jogo de ação 2D com combate em tempo real.
                                \item Quebra-cabeça desafiador, como a montagem do Cavalo de Tróia.
                                \item Batalhas épicas, incluindo a luta de Heitor contra Menelau.
                                \item Decisão sobre o destino de Tróia: será que ela cairá ou sobreviverá?
                            \end{itemize}
                        \end{block}
                        
                        % TÍTULO TECNOLOGIAS UTILIZADAS
                        \begin{block}{\textbf{Tecnologias Utilizadas}}
                            \begin{itemize}
                                \item Linguagem: C
                                \item Ambiente de Desenvolvimento: Visual Studio 2022
                                \item Biblioteca: Allegro 5
                                \item Design: Figma e Canva
                                \item Arte: Sprites 2D (criação utilizando o repositório \href{https://sanderfrenken.github.io/Universal-LPC-Spritesheet-Character-Generator/}{Universal LPC Spritesheet Generator})

                            \end{itemize}
                        \end{block}

                        % TÍTULO IMAGEM DO JOGO
                        \begin{block}{\textbf{Imagens do Jogo}}
                            \begin{minipage}{0.48\textwidth}
                                \centering
                                \includegraphics[width=\textwidth]{menu.png} 
                                \vspace{0.3cm}
                                \includegraphics[width=\textwidth]{intro.png} 
                            \end{minipage}%
                            \hfill
                            \begin{minipage}{0.48\textwidth}
                                \centering
                                \includegraphics[width=\textwidth]{fase2.jpg} 
                                \vspace{0.3cm}
                                \includegraphics[width=\textwidth]{puzzles.png}
                            \end{minipage}
                            \vspace{0.3cm}
                            \begin{minipage}{0.48\textwidth}
                                \centering
                                \includegraphics[width=\textwidth]{fase4.jpg} 
                            \end{minipage}
                        \end{block}
                        
                        \vfill
                    }
                \end{minipage}
            \end{beamercolorbox}
        \end{column}
        % ---------------------------------------------------------%
    \end{columns}
    \vskip
\end{frame}
\end{document}
